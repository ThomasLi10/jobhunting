
% -------------------- Environment Setup -------------------- 
\documentclass[titlepage=false]{ctexart}
\CTEXsetup[format={\Large\bfseries}]{section}
\bibliographystyle{plain}
\usepackage[left=2.00cm, right=2.00cm, top=2.00cm, bottom=2.00cm]{geometry}
\usepackage{amsmath,amsfonts,amssymb,amsbsy} % math equations, symbols
\usepackage{fancyhdr}   % 设置页眉、页脚
\usepackage{hyperref}
\usepackage{color}
\usepackage{algorithm}
\usepackage{algorithmic}
\renewcommand{\algorithmicrequire}{ \textbf{Input:}}     % use Input in the format of Algorithm  
\renewcommand{\algorithmicensure}{ \textbf{Initialize:}} % use Initialize in the format of Algorithm  
\renewcommand{\algorithmicreturn}{ \textbf{Output:}}     % use Output in the format of Algorithm  

\usepackage{inconsolata}
\fontspec{inconsolata}
\setmonofont[StylisticSet=1]{inconsolata}

\usepackage{xcolor}
\usepackage{listings}
\lstset{basicstyle=\ttfamily,
showstringspaces=false,
commentstyle=\color{red},
keywordstyle=\color{blue}
}

% -------------------- Title -------------------- 
\title{\textbf{凯读笔试题目程序文档}}
\author{李想}
\date{\today}

% -------------------- Document -------------------- 
\pagestyle{fancy}
\begin{document}
\maketitle

% Contents
\section{程序运行} 

核心类 \texttt{MovingAverage} 位于脚本 \texttt{kdma.py} 中。 

脚本 \texttt{test.py} 中构造了一系列间隔不固定的时间戳和价格序列,用于测试该类。

请直接运行 
\begin{lstlisting}[language=bash]
  $ python test.py
\end{lstlisting}
测试代码。


\section{Notation} 

$t_i$:第$i$个数据点的时间戳

$p_i$:第$i$个数据点的价格

$\tau_i$:第$i$个数据点与上个数据点的时间间隔

$W$:窗口长度

$w_{i|t=t_N}$:计算时间戳为$t_N$的数据点到达后的平均价格时,第$i$个数据点的权重


$\bar{p}_i$:第$i$个数据点到达后的平均价格


% 平均价格算法
\section{平均价格} 

\subsection{数学表达} 
为使平均价格应能较好反映股票价格在窗口时间内的“平均水平”,程序中使用每个价格对应时间戳距离最新时间戳的距离为权重,计算平均价格。

记某个时刻共到达$N$条数据,时间戳与价格的数据对为 $\{(t_i,p_i)\}, i=1,2,\cdots,N$,
则最新到达的数据时间戳为 $t_N$。窗口时间长度为 $W$,最新窗口期$[t_N-W,W]$内共有 $K(K\ge 1)$ 条数据,
数据对为$\{(t_i,p_i)\}, i=N-K+1,\dots,N$。

在计算平均价格时,时刻$t_i$的权重 
\begin{equation}\label{eq:wt}
    w_{i|t=t_N}=W-(t_N-t_i)=W+t_i-t_N,\quad i=N-K+1,\dots,N
\end{equation}

时刻$t_N$的平均价格则可以表示为
\begin{equation}\label{eq:avg_price}
    \bar{p}_N = \frac{\sum_{i=N-K+1}^N{p_iw_{i|t=t_N}}}{\sum_{i=N-K+1}^N{w_{i|t=t_N}}}
\end{equation}

\subsection{权重更新}\label{ssec:wt_udpate_without_limit}
类\texttt{MovingAverage}中对权重是这样处理的:在暂时不考虑内存限制的情况(考虑内存限制下的权重更新方法参见\ref{ssec:wt_udpate_with_limit})下,在$t_N$时刻,第$i$条数据的权重
\begin{equation}\label{eq:wt_update_N}
    w_{i|t=t_N} = W+t_i-t_N
\end{equation}

当$t_{N+1}$时刻的数据到达时,该权重应被更新为
\begin{align}\label{eq:wt_update_N+1}
    w_{i|t=t_{N+1}} &= W+t_i-t_{N+1} = W+t_i-t_N-(t_{N+1}-t_N)\\
                    &=w_{i|t=t_N} - \tau_{N+1}
\end{align}
其中 $\tau_{N+1}=t_{N+1}-t_N$ 为两个时刻的时间间隔。



\subsection{鲁棒性} 

由于相邻数据点之间的间隔不固定,所以在窗口期$[t_N-W,W]$内,可能存在多个数据点,也可能只有一个数据点(即最新的数据$\{(t_N,p_N)\}$)。
式\eqref{eq:avg_price}对上述两种情况无差别对待,均可计算出合理的平均价格。



% 不考虑内存
\section{不考虑内存限制的算法}

当内存没有任何限制时,我们可以将窗口期内任意多的数据点都记录下来,可以精确地计算式\eqref{eq:avg_price}所表达的平均价格。
具体算法见算法\ref{algo:nomemlimit}:


\begin{algorithm}
	\caption{不考虑内存限制的算法}
	\label{algo:nomemlimit}
	\begin{algorithmic}[1]
        \REQUIRE ~\\
            时间戳与价格的数据对时间序列$\{(t_i,p_i)\},i=1,2,\dots,N$.\\  % this command shows "Input"
            窗口期长度 $W$
		\ENSURE ~\\           % this command shows "Initialized"
            最新数据的时间戳 $t=0$\\
            窗口期内数据点的个数 $n=0$\\
            窗口期内数据点的时间戳列表 ${\rm L}_t=[\,]$\\
            窗口期内数据点的时间间隔列表 ${\rm L}_{\tau}=[\,]$\\
            窗口期内数据点的时间间隔之和 $S_{\tau}=0$\\
            窗口期内数据点的价格列表 ${\rm L}_p=[\,]$\\
            窗口期内数据点的权重列表 ${\rm L}_w=[\,]$\\
        \FOR{$i$ in $\{1,2,\dots,N\}$}
        \STATE $\tau=t-t_i,\;t=t_i$
        \WHILE { $n>0$ and $S_{\tau}+\tau>W$}
            \STATE 按先进先出法弹出${\rm L}_{t},{\rm L}_{\tau},{\rm L}_{p},{\rm L}_{w}$各列表中的第一个元素,记为$t_0,\tau_0,p_0,w_0$
            \STATE $S_{\tau} = S_{\tau}-\tau_0$
            \STATE $n = n-1$
		\ENDWHILE
        \STATE 将 ${\rm L}_{w}$ 中各元素分别减去 $\tau$
        \STATE 将 $t_i,\tau,p_i,W$ 分别存入${\rm L}_{t},{\rm L}_{\tau},{\rm L}_{p},{\rm L}_{w}$各列表尾部
        \STATE $S_{\tau} = S_{\tau}+\tau$
        \STATE $n = n+1$
        \STATE $\bar{p}_i={{\rm sum}({\rm L}_{p}\cdot{\rm L}_{w})}/{{\rm sum}({\rm L}_{w})}$
        \ENDFOR
        \RETURN 平均价格序列$\{\bar{p}_i\},i=1,2,\dots,N$
	\end{algorithmic}
\end{algorithm}


\section{考虑内存限制的算法}

当内存存在限制时,可能会出现无法将窗口期内所有数据点都记录下来的情况,这时式\eqref{eq:avg_price}将无法直接适用。

解决方案是:\textbf{当最新数据到达时,在计算最新平均价格之前,若最新窗口期内的数据点数量已经达到内存上限时,找出上一个窗口期内所有数据点中时间间隔最小的两个数据点,将二者的信息“合并”存储,再将最新数据存入,计算最新平均价格}。

\subsection{“合并”存储}

对于任意两个相邻的数据点$(t_{i-1},\tau_{i-1},p_{i-1},w_{i-1}),(t_{i},\tau_{i},p_{i},w_{i})$,合并后的权重为二者权重之和,价格为二者价格的加权平均,且合并后沿用后一时刻的时间数据(可以理解为把“合并”后的数据存储在后一时刻)
\begin{align}\label{eq:merge}
    t_i' &= t_i\\
    \tau_i' &= \tau_{i-1} + \tau_{i}\\
    w_i' &= w_{i-1} + w_i\\
    p_i' &= \frac{p_{i-1}w_{i-1} + p_iw_i}{w_i'}
\end{align}

为什么选择把“合并”后的数据点存储在后一时刻而非前一时刻或中间某时刻,是考虑到若不存储在后一时刻,那么当窗口向后滚动时,很有可能会将这个“合并”的数据点直接排除在新窗口之外,丢失信息导致计算偏差变大。



如此,如果这两个数据点的时间间隔足够小,则可以将“合并”后的数据点视为一个正常数据点,用于计算平均价格。

同理,还可以会出现需要“合并”已被“合并”数据点的情况,但只要是两个数据点间隔足够小,近似程度仍会比较高。



\subsection{内存限制时权重的更新}\label{ssec:wt_udpate_with_limit}

\ref{ssec:wt_udpate_without_limit}部分介绍了在没有内存限制时如何在新数据到达时对权重进行更新。
当存在内存限制时,因为要“合并”数据,所以对已经被“合并”的数据,式\eqref{eq:wt_update_N+1}不再适用。
但此式仍可以给我们一些启发:如果某个数据点中的信息只有一个原始数据点,那么更新权重时要减去一个$\tau$。
如果某个“合并”数据点中存有$k$个原始数据点的信息,那么更新权重时只要减去$k$个$\tau$就可以了。

按上述思路,在循环计算平均价格过程中还需要记录“合并”次数这一信息。

\subsection{算法}

考虑有内存限制时的算法见算法\ref{algo:memlimit}:

\begin{algorithm}
	\caption{考虑内存限制的算法}
	\label{algo:memlimit}
	\begin{algorithmic}[1]
        \REQUIRE ~\\
            时间戳与价格的数据对时间序列$\{(t_i,p_i)\},i=1,2,\dots,N$.\\  % this command shows "Input"
            窗口期长度 $W$\\
            内存上限 $n_B$
		\ENSURE ~\\           % this command shows "Initialized"
            最新数据的时间戳 $t=0$\\
            窗口期内数据点的个数 $n=0$\\
            窗口期内数据点的时间戳列表 ${\rm L}_t=[\,]$\\
            窗口期内数据点的时间间隔列表 ${\rm L}_{\tau}=[\,]$\\
            窗口期内数据点的时间间隔之和 $S_{\tau}=0$\\
            窗口期内数据点的价格列表 ${\rm L}_p=[\,]$\\
            窗口期内数据点的权重列表 ${\rm L}_w=[\,]$\\
            窗口期内数据点的“合并”次数列表 ${\rm L}_c=[\,]$\\
        \FOR{$i$ in $\{1,2,\dots,N\}$}
        \STATE $\tau=t-t_i,\;t=t_i$
        \WHILE { $n>0$ and $S_{\tau}+\tau>W$}
            \STATE 按先进先出法弹出${\rm L}_{t},{\rm L}_{\tau},{\rm L}_{p},{\rm L}_{w},{\rm L}_{c}$各列表中的第一个元素,记为$t_0,\tau_0,p_0,w_0,c_0$
            \STATE $S_{\tau} = S_{\tau}-\tau_0$
            \STATE $n = n-1$
		\ENDWHILE
        \IF {$n=n_B$}
            \STATE $j = ({\rm argmin}\; {\rm L}_{\tau})+1$
            \STATE 弹出${\rm L}_{t},{\rm L}_{\tau},{\rm L}_{p},{\rm L}_{w},{\rm L}_{c}$各列表中的第$j$个元素,记为$t_j,\tau_j,p_j,w_j,c_j$
            \STATE $n=n-1$
            \STATE ${\rm L}_{t}[j-1] = t_j$
            \STATE ${\rm L}_{\tau}[j-1] = {\rm L}_{\tau}[j-1]+\tau_j$
            \STATE $w' = {\rm L}_{w}[j-1]+w_j$
            \STATE $p' = \frac{ {\rm L}_{p}[j-1]{\rm L}_{w}[j-1]+ p_jw_j}{w'}$
            \STATE ${\rm L}_{w}[j-1] = w'$
            \STATE ${\rm L}_{p}[j-1] = p'$
            \STATE ${\rm L}_{c}[j-1] = {\rm L}_{c}[j-1]+c_j$
        \ENDIF
        \STATE ${\rm L}_{w}[k] = {\rm L}_{w}[k]-\tau{\rm L}_{c}[k],k=1,2,\dots,i$
        \STATE 将 $t_i,\tau,p_i,W,1$ 分别存入${\rm L}_{t},{\rm L}_{\tau},{\rm L}_{p},{\rm L}_{w},{\rm L}_{c}$各列表尾部
        \STATE $S_{\tau} = S_{\tau}+\tau$
        \STATE $n = n+1$
        \STATE $\bar{p}_i={{\rm sum}({\rm L}_{p}\cdot{\rm L}_{w})}/{{\rm sum}({\rm L}_{w})}$
        \ENDFOR
        \RETURN 平均价格序列$\{\bar{p}_i\},i=1,2,\dots,N$
	\end{algorithmic}
\end{algorithm}











\end{document}
